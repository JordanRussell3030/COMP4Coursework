\chapter{Analysis}

\section{Introduction}

\subsection{Client Identification}

The client for this project is Trish Marshall, a maths teacher at a secondary school who requires a more up-to-date program to help teach her students trigonometry based on next years new GCSE curriculum.

\subsection{Define the current system}

The current system is a website called MyMaths which is used to teach all areas of maths from SAT level to A-Level. It consists of lessons which are interactive and give problems with solutions, followed by homework tasks which can be set by the teacher. The status of these tests and the results are recorded for each student. The client also uses a smart board to demonstrate methods, and gives out textbooks to be read.

\subsection{Describe the problems}

The main problem with the current system is that it is not designed in a way that sufficiently challenges the student, for example, the button for the answer to a question will have the answer to the next question behind it, which presents a risk of the student accidentally double clicking and getting a lucky mark, or in some cases, they are just lazy and assume it's right. Furthermore, the sine, cosine and tan buttons are always put in that order, which minimises the amount if thinking a student will have to do to figure out which one is right for the problem they are solving, otherwise they get used to it already being in the same place for them.

In some of the examples used on the website MyMaths, the working out shows unnecessary stages which could sometimes put students off getting the right answer, such as when it gives an inaccurately rounded decimal number to represent a fraction when only the fraction is needed to find the solution.

MyMaths does not include a section with problems for any of the three rules, which limits the students ability to work out which rule will be needed, if all of the problems in one test use the same rule. Alongside this problem, MyMaths does not have a means of teaching the student how to know whether a problem will use the sine rule or the cosine rule. It also doesn't teach the sine, cosine and tan rules all together.

The current website is not designed to completely support the new GCSE maths curriculum which will be implemented starting from the 2016/2017 academic year, so will only be up to date for one more year, unless they change it.

The lessons on MyMaths teach the students how to calculate angles before how to calculate sides, which is a problem because in most cases you have to be able to use the inverse function of the rule to find the missing angle, for which you need to know which side is missing or which sides are in use, and what their values are.

The only wy of inputting answers is by typing into boxes, which can become repetitive and boring.

The feedback system is not visual enough or convenient enough to use effectively, for example, there is no quick way of instructing a student to a detention or a meeting with their teacher.

\subsection{Section appendix}

\section{Investigation}

\subsection{The current system}

\subsubsection{Data sources and destinations}

\subsubsection{Algorithms}

\subsubsection{Data flow diagram}

\subsubsection{Input Forms, Output Forms, Report Formats}

\subsection{The proposed system}

\subsubsection{Data sources and destinations}

\subsubsection{Data flow diagram}

\subsubsection{Data dictionary}

\subsubsection{Volumetrics}

\section{Objectives}

\subsection{General Objectives}

\subsection{Specific Objectives}

\subsection{Core Objectives}

\subsection{Other Objectives}

\section{ER Diagrams and Descriptions}

\subsection{ER Diagram}

\subsection{Entity Descriptions}

\section{Object Analysis}

\subsection{Object Listing}

\subsection{Relationship diagrams}

\subsection{Class definitions}

\section{Other Abstractions and Graphs}

\section{Constraints}

\subsection{Hardware}

\subsection{Software}

\subsection{Time}

\subsection{User Knowledge}

\subsection{Access restrictions}

\section{Limitations}

\subsection{Areas which will not be included in computerisation}

\subsection{Areas considered for future computerisation}

\section{Solutions}

\subsection{Alternative solutions}

\subsection{Justification of chosen solution}