\chapter{Analysis}

\section{Introduction}

\subsection{Client Identification}

The client for this project is Trish Marshall, who is a maths teacher at a secondary school who teaches all variations of maths in the current curriculum to GCSE students. She finds that the trigonometry resources she currently has are not as effective for teaching as she would like. The curriculum is changing from the 2016-17 academic year onwards, so the current resources are not up to date. She would like this program to enable her to use more up-to-date methods of teaching trigonometry to support the new curriculum.

\subsection{Define the current system}

The current system is a website called MyMaths which is used to teach all areas of maths from SAT level to A-Level. It is accessible online by anyone who signs up through a centre of education, for example a maths class. It consists of lessons which are interactive and give problems with solutions, followed by homework tasks which can be set by the teacher. The status of these tests and the results are recorded for each student in an online database accessible by the teacher. Homeworks can be set online and progress is recorded so the teacher can view the submissions from the students and take appropriate action following the deadline. The client also uses a smart board to demonstrate methods, and gives out textbooks to be read.

\subsection{Describe the problems}

The main problem with the current system is that it is not designed in a way that sufficiently challenges the student, for example, the button for the answer to a question will have the answer to the next question behind it, which presents a risk of the student accidentally double clicking and getting a lucky mark, or in some cases, they are just lazy and assume it's right. Furthermore, the sine, cosine and tan buttons are always put in that order, which minimises the amount if thinking a student will have to do to figure out which one is right for the problem they are solving, otherwise they get used to it already being in the same place for them.

In some of the examples used on the website MyMaths, the working out shows unnecessary stages which could sometimes put students off getting the right answer, such as when it gives an inaccurately rounded decimal number to represent a fraction when only the fraction is needed to find the solution.

MyMaths does not include a section with problems for any of the three rules, which limits the students ability to work out which rule will be needed, if all of the problems in one test use the same rule. Alongside this problem, MyMaths does not have a means of teaching the student how to know whether a problem will use the sine rule or the cosine rule. It also doesn't teach the sine, cosine and tan rules all together.

The current website is not designed to completely support the new GCSE maths curriculum which will be implemented starting from the 2016/2017 academic year, so will only be up to date for one more year, unless they change it.

The lessons on MyMaths teach the students how to calculate angles before how to calculate sides, which is a problem because in most cases you have to be able to use the inverse function of the rule to find the missing angle, for which you need to know which side is missing or which sides are in use, and what their values are.

The only wy of inputting answers is by typing into boxes, which can become repetitive and boring.

The feedback system is not visual enough or convenient enough to use effectively, for example, there is no quick way of instructing a student to a detention or a meeting with their teacher.

\subsection{Section appendix}

The client interview was conducted by meeting in person and asking the questions directly, with access to the current system to show in detail precisely what the problems were, and what she needed improving upon. An up to date textbook was also present for referencing. The questionnaire was as follows:

1. What do you require the proposed system to do?

2. What are the problems with the current way of doing things?

3. What data or information is recorded in the current system?

4. Is the any data or information you require to be recorded in the proposed system? If so, how much?

5. If there is data, how frequently will it need to be updated?

6. Will new records need to be added and old ones deleted?

7. How important is the data or information that is recorded?

8. What processes or functions are performed by the current system?

9. What processes and functions are to be performed by the new system?

10. What special algorithms do these processes use?

11. Which processes should be executed manually?

12. What are the inputs to the current system?

13. What inputs will be required fo rthe proposed system?

14. What are the outputs from the current system?

15. What outputs will be required from the proposed system?

16. Are hard copies required?

17. What computing resources does the client possess?

18. Are there any security issues?

19. Should there be restricted access to particular areas?

20. how are exceptions and errors handled in the current system?

21. What errors and exceptions should be reported in the proposed system?

22. How should they be reported?

23. Are there any constraints on hardware, software, data, methods of working, cost, time, etc?

24. Does the user have a particular solution in mind?

\section{Investigation}

\subsection{The current system}

\subsubsection{Data sources and destinations}

\subsubsection{Algorithms}

\subsubsection{Data flow diagram}

\subsubsection{Input Forms, Output Forms, Report Formats}

\subsection{The proposed system}

\subsubsection{Data sources and destinations}

\subsubsection{Data flow diagram}

\subsubsection{Data dictionary}

\subsubsection{Volumetrics}

\section{Objectives}

\subsection{General Objectives}

\subsection{Specific Objectives}

\subsection{Core Objectives}

\subsection{Other Objectives}

\section{ER Diagrams and Descriptions}

\subsection{ER Diagram}

\subsection{Entity Descriptions}

\section{Object Analysis}

\subsection{Object Listing}

\subsection{Relationship diagrams}

\subsection{Class definitions}

\section{Other Abstractions and Graphs}

\section{Constraints}

\subsection{Hardware}

\subsection{Software}

\subsection{Time}

\subsection{User Knowledge}

\subsection{Access restrictions}

\section{Limitations}

\subsection{Areas which will not be included in computerisation}

\subsection{Areas considered for future computerisation}

\section{Solutions}

\subsection{Alternative solutions}

\subsection{Justification of chosen solution}