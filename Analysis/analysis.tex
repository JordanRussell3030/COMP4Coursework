\chapter{Analysis}

\section{Introduction}

\subsection{Client Identification}

The client for this project is Trish Marshall, who is a maths teacher at a secondary school who teaches all variations of maths in the current curriculum to GCSE students. She finds that the trigonometry resources she currently has are not as effective for teaching as she would like. The curriculum is changing from the 2016-17 academic year onwards, so the current resources are not up to date. She would like this program to enable her to use more up-to-date methods of teaching trigonometry to support the new curriculum.

\subsection{Define the current system}

The current system is a website called MyMaths which is used to teach all areas of maths from SAT level to A-Level. It is accessible online by anyone who signs up through a centre of education, for example a maths class. It consists of lessons which are interactive and give problems with solutions, followed by homework tasks which can be set by the teacher. The status of these tests and the results are recorded for each student in an online database accessible by the teacher. Homeworks can be set online and progress is recorded so the teacher can view the submissions from the students and take appropriate action following the deadline. The client also uses a smart board to demonstrate methods, and gives out textbooks to be read.

\subsection{Describe the problems}

The main problem with the current system is that it is not designed in a way that sufficiently challenges the student, for example, the button for the answer to a question will have the answer to the next question behind it, which presents a risk of the student accidentally double clicking and getting a lucky mark, or in some cases, they are just lazy and assume it's right. Furthermore, the sine, cosine and tan buttons are always put in that order, which minimises the amount if thinking a student will have to do to figure out which one is right for the problem they are solving, otherwise they get used to it already being in the same place for them.

In some of the examples used on the website MyMaths, the working out shows unnecessary stages which could sometimes put students off getting the right answer, such as when it gives an inaccurately rounded decimal number to represent a fraction when only the fraction is needed to find the solution.

MyMaths does not include a section with problems for any of the three rules, which limits the students ability to work out which rule will be needed, if all of the problems in one test use the same rule. Alongside this problem, MyMaths does not have a means of teaching the student how to know whether a problem will use the sine rule or the cosine rule. It also doesn't teach the sine, cosine and tan rules all together.

The current website is not designed to completely support the new GCSE maths curriculum which will be implemented starting from the 2016/2017 academic year, so will only be up to date for one more year, unless they change it.

The lessons on MyMaths teach the students how to calculate angles before how to calculate sides, which is a problem because in most cases you have to be able to use the inverse function of the rule to find the missing angle, for which you need to know which side is missing or which sides are in use, and what their values are.

The only wy of inputting answers is by typing into boxes, which can become repetitive and boring.

The feedback system is not visual enough or convenient enough to use effectively, for example, there is no quick way of instructing a student to a detention or a meeting with their teacher.

\subsection{Section appendix}

The client interview was conducted by meeting in person and asking the questions directly, with access to the current system to show in detail precisely what the problems were, and what she needed improving upon. An up to date textbook was also present for referencing. The questionnaire was as follows:

1. What do you require the proposed system to do?

The proposed system needs to interactively teach the updated GCSE trigonometry syllabus in a way that reduces the number of flaws compared to the current system. It needs to have a range of difficulty for the questions to teach a range of abilities, and it needs to cover every part of GCSE trigonometry in the most effective order. It must also provide an adequate amount of homework tasks to properly test their knowledge following the lessons, and these tasks should be submitted upon completion to be viewed by the teacher. It must also be quick and easy for a teacher to view the students progress with trigonometry, mostly by assessing their homework, with feedback and warnings being easy to give out when required. Finally, it must save a record of every task set for every student, and its completion progress and results, and these records need to be easily accessible. Every student and teacher needs their own individual login to avoid shared or inaccurate progression.

2. What are the problems with the current way of doing things?

Some of the biggest problems are with the way in which things are taught on the current system; in some cases, the method demonstrations are inaccurate and can mislead students away from the correct solution. Furthermore, the button for two consecutive answers are often in the same place, which can allow a student to get the right answer by assuming the answer is in that place after this problem occurring enough times prior to that task, or even by just double-clicking by accident. Answer buttons should be jumbled up, as this problem can be prevented, and can also solve the problem of the students not properly distinguishing the differences between the sine, cosine and tan rules, as they always appear in the same order and are taken for granted. The order in which lessons are taught is also problematic as sides should be taught before angles to allow for the student to learn how to use the inverse function of the rule which is often a key part of finding a missing angle, whereas MyMaths teaches angles before sides. Alongside this problem is that the sine, cosine and tan rules are all taught individually. If they were taught together it would give the student more experience in deciding which rule they actually have to use, rather than just being told. MyMaths also lacks a lesson on how to tell the whether the problem will use the sine or cosine rule. Lastly, the current system does not cover the new GCSE curriculum which will be implemented next year (2016-17), so more focus on the problem-solving aspect is required.

3. What data or information is recorded in the current system?

The currently recorded fields in the homework set table are: Name (Text), Task or activity (Text), Type (Text), Created (Date), Completed (Text), Start (Date, Foreign Key), Due (Date), Feedback (Text, Foreign Key).

The currently recorded fields in the results table are: Level (Integer), Topic (Text), Task name (Text), Number of tries (Integer), Start (Date, Foreign Key), Last tried (Date), Rating (Integer), Percentage (Real), Question number percentage (Real), Feedback (Text, Foreign Key).

The currently recorded field in the administration table is: Students belong to these classes (Text).

4. Is the any data or information you require to be recorded in the proposed system? If so, how much?

I require the following data to be recorded in the proposed system: User ID (Integer, Primary Key), First name (Text), Surname (Text), Overall percentage score (Real), Red/Amber/Green face (Graphic, Text, Foreign Key), Feedback (Text).

Also, in a separate table I need a field for each question to view the individual percentage so I know exactly what the student needs to work on, along with the feedback, red/amber/green face and whether or not the student needs to see me after class. 

Example: Question 1 percentage (Real), Question 2 percentage (Real), Question 3 - x (Real), Red/Amber/Green face (Graphic, Text, Foreign Key), Appointment with teacher? (Text, Y/N), Time (Time, only if Appointment with teacher is Yes)

5. If there is data, how frequently will it need to be updated?

The data will need to be updated every time a student submits a finished or amended homework assignment.

6. Will new records need to be added and old ones deleted?

The old records will not need to be deleted, and new ones will be added as described above.

7. How important is the data or information that is recorded?

It is very important as it allows both teacher and student to track progress and know what needs improving upon, and allows the teacher to record evidence of the amount of work and revision each student has done. However progress will not be actively monitored, only checked following deadlines/

8. What processes or functions are performed by the current system?

The current system saves data to the database which can be viewed by both student and teacher. It uses an interactive GUI so that the user can navigate between many pages and areas of the website, and practice tasks and homework tasks use text boxes for the user to fill in to submit an answer.

9. What processes and functions are to be performed by the new system?

The proposed system must also save data to a database, but it must be much easier to access, navigate and use generally. The program will be navigable using a GUI, and lessons, homework tasks and the database will be accessible. The lessons and homeworks will be interactive and use text boxes for answers, as well as drag and drop graphics and boxes for showing working out.

10. What special algorithms do these processes use?

To save data to the database, the current system uses a read/write algorithm, writing to store, and reading to view in the graphical database. For navigating the pages and submitting answers, there are many radio buttons on the website which will all use the same connection function which is built in to Python. The test questions themselves will use selection statements to determine if a submitted answer is the same as the expected correct answer, stored as a fixed variable. To log on to the website, the program uses a validation algorithm to check the user name and password is correct, and uses error exception algorithms to ask the user for the correct input if necessary.

11. Which processes should be executed manually?

Clicking the connection radio buttons, submitting answers, logging on, navigating the GUI.

12. What are the inputs to the current system?

User name and password, cursor clicks for navigation and selecting text boxes, keyboards for typing answers, cursor clicks for submitting answers.

13. What inputs will be required for the proposed system?

Cursor clicks for navigation, selecting text boxes, submitting answers and dragging graphics, and keyboard for typing answers, user name and password.

14. What are the outputs from the current system?

It outputs an error handling exception if the user name or password is incorrect, a message appears asking you to re-enter the correct inputs. It outputs a tick or a cross beside submitted answers, and graphics to represent how well you did on the database, such as a red or green face. It outputs your score percentage, your rating, and your feedback from the teacher. 

15. What outputs will be required from the proposed system?

The system is required to display methods of working out next to an incorrect answer submission, rather than just a big cross. If a student inputs the wrong data type, it will produce an error exception hint saying "Please input an integer" or "Please input a decimal" or "Please input an angle". It will give a hint in the same way if a student gets the wrong answer using the correct data type, then when they get it wrong the second time it will display the answer as well as the method of working out, and give them marks for their working out accordingly rather than the big cross. The data stored in the database will need to be easily readable and the graphics should be appropriate.

16. What computing resources does the client possess?

The centre has many computer rooms, a smart board in the classroom, and some students can gain permission to bring in their own devices.

17. Are there any security issues?

I may have to gain permission from IT if you need to install your program on the school computers for testing on a class, but this shouldn't be an issue if you are trustworthy and your program can be scanned for insecurities.

18. Should there be restricted access to particular areas?

Students should not be allowed to view the progress of the whole class, only teachers. Teachers should also have exclusive access to manually setting homeworks and sending personal messages or warnings to the students. 

19. how are exceptions and errors handled in the current system?

When logging in, if the user name or password is invalid it asks you to input the correct values. If an answer is wrong, or uses the wrong data type, it just takes off the marks and moves on.

20. What errors and exceptions should be reported in the proposed system?

If a user name or password is incorrect, a small box should pop up asking for the correct value. If a wrong data type is used to answer a question, a small box should pop up asking for the correct data type (e.g. "Please input an integer, decimal or angle") and will not dock marks or remaining attempts. If a student gets it wrong on their first attempt, a small box should pop up giving a hint, and dock one of the two attempts. If they get it wrong again, the correct answer along with the full method for working out will appear next to the box.

21. How should they be reported?

Small windows temporarily popping up on screen with messages, until the OK button or the cross in the top right is clicked.

22. Are there any constraints on hardware, software, data, methods of working, cost, time, etc?

The hardware and software accessible by the client are sufficient. The data should be saved in the database successfully. Methods of working are usual, just navigating the program and inputting the answers. There is no cost for this program. There may be time constraints if difficulties are encountered during the development process.

\section{Investigation}

\subsection{The current system}

\subsubsection{Data sources and destinations}

\subsubsection{Algorithms}

\subsubsection{Data flow diagram}

\subsubsection{Input Forms, Output Forms, Report Formats}

\subsection{The proposed system}

\subsubsection{Data sources and destinations}

\subsubsection{Data flow diagram}

\subsubsection{Data dictionary}

\subsubsection{Volumetrics}

\section{Objectives}

\subsection{General Objectives}

\subsection{Specific Objectives}

\subsection{Core Objectives}

\subsection{Other Objectives}

\section{ER Diagrams and Descriptions}

\subsection{ER Diagram}

\subsection{Entity Descriptions}

\section{Object Analysis}

\subsection{Object Listing}

\subsection{Relationship diagrams}

\subsection{Class definitions}

\section{Other Abstractions and Graphs}

\section{Constraints}

\subsection{Hardware}

\subsection{Software}

\subsection{Time}

\subsection{User Knowledge}

\subsection{Access restrictions}

\section{Limitations}

\subsection{Areas which will not be included in computerisation}

\subsection{Areas considered for future computerisation}

\section{Solutions}

\subsection{Alternative solutions}

\subsection{Justification of chosen solution}